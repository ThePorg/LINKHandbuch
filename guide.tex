\documentclass[titlepage,a4paper]{article}

\usepackage[utf8]{inputenc}
\usepackage{makeidx}
\usepackage{graphicx}
\graphicspath{{figures/}}
\usepackage{amsmath}
\usepackage{theorem}
\usepackage[german]{babel}
\usepackage{cite}
\usepackage{textcomp}
\usepackage{fancyhdr}
\usepackage{float}
\usepackage{listings}
\usepackage{subcaption}
\fancyhead{}
\fancyhead[LO]{\bfseries \rightmark}
\fancyfoot[LO]{\thepage}
\fancyfoot[CO]{}
\pagestyle{fancy}
\bibliographystyle{ieeetr}
\usepackage[title]{appendix}
\linespread{1.3}


\title{Ausstellung "Link zur K.I."\\
Handbuch Interaktive Stationen\\
(WIP)}
\author{
    AG Reiterer\\
    Universität Konstanz
}

\date{September 2019}

\makeindex



\begin{document}

\begin{titlepage}
\maketitle
\end{titlepage}


\pagenumbering{roman}

\tableofcontents

\pagebreak

\listoffigures 

\pagebreak

\pagenumbering{arabic}

\section{Allgemeines}

Alle Computer haben das Passwort \textit{link}.

\section{Raum 1 ('Desktop')}

\subsection{Videoinstallation}

\subsubsection{Hardware}

\begin{itemize}
\item 3 Canon XEED WUX500ST Projektoren
\item 2 BenQ Projektoren
\item 1 Medienrechner
\item 1 Bildschirm
\end{itemize}

\subsubsection{Aufbau}

Die Canon-Beamer werden verwendet um die Projektionsflächen im Raum zu bespielen, die BenQ-Beamer für die Rückwand. Die Beamer sind gemeinsam mit einem Bildschirm zur Steuerung an dem Medienrechner angeschlossen, wober der Steuerungsbildschirm an die Grafikkarte angeschlossen werden muss.

(Abb. Video-Ins)

\subsubsection{Software}

Der Medienrechner steuert die Projektoren mit dem Programm Resolume an. Um die Wiedergabe zu starten muss in diesem ein Doppelklick auf dem Reiter über den Video-Vorschaubildern ausgeführt werden.

(Abb. Resolume)

\section{Raum 2 ('Platine')}

\section{Raum 3 ('Server')}

\section{Raum 4 ('Cloud')}

\subsection{Chatbot 'Aski'}

\subsubsection{Hardware}

\begin{itemize}
\item 5 Surface Pro 6
\item 5 4k-Fernseher
\item USB-Schlösser
\end{itemize}

\subsubsection{Aufbau}

Die Fernseher und Tablets werden in den Seilen als Paare montiert und per Videokabel (Mini-Display Port auf HDMI, 4k-fähig) verbunden. Die USB-Slots werden mit dem beiliegenden Schloss-System vor ungewünschten Zugriffen geschützt.

\subsubsection{Software}

Die Surfaces stellen ihren Inhalt selbst über einen npm-Server bereit. Dieser kann auf dem Desktop mit einem Rechtsklick $\rightarrow$ 'Mit Windows Powershell ausführen' auf die Datei 'Starte Aski' gestartet werden. Der Inhalt kann dann mit Chrome angezeigt werden, wobei Chrome über die Verknüpfung auf dem Desktop gestartet werden sollte. Dabei muss eine Maus an das Surface angeschlossen sein, um das Anzeigefenster für Chat und Video auf den Fernseher zu bewegen.

\end{document}