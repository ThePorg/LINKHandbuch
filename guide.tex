\documentclass[titlepage,a4paper]{article}

\usepackage[utf8]{inputenc}
\usepackage{makeidx}
\usepackage{graphicx}
\graphicspath{{figures/}}
\usepackage{amsmath}
\usepackage{theorem}
\usepackage[german]{babel}
\usepackage{cite}
\usepackage{textcomp}
\usepackage{fancyhdr}
\usepackage{float}
\usepackage{listings}
\usepackage{subcaption}
\fancyhead{}
\fancyhead[LO,RE]{\bfseries \rightmark}
\fancyfoot[LO,RE]{\thepage}
\fancyfoot[CE,CO]{}
\pagestyle{fancy}
\bibliographystyle{ieeetr}
\usepackage[title]{appendix}
\linespread{1.3}


\title{Ausstellung "Link zur K.I."\\
Handbuch Interaktive Stationen}
\author{
    AG Reiterer\\
    Universität Konstanz
}

\date{August 2019}

\makeindex



\begin{document}

\begin{titlepage}
\maketitle
\end{titlepage}


\pagenumbering{roman}

\tableofcontents

\pagebreak

\listoffigures 

\pagebreak

\pagenumbering{arabic}

\section{Raum 1 ('Desktop')}

\subsection{Videoinstallation}

\subsubsection{Hardware}

\begin{itemize}
\item 3 Canon XEED WUX500ST Projektoren
\item 2 BenQ Projektoren
\item 1 Medienrechner
\item 1 Bildschirm
\end{itemize}

\subsubsection{Aufbau}

Die Canon-Beamer werden verwendet um die Projektionsflächen im Raum zu bespielen, die BenQ-Beamer für die Rückwand. Die Beamer sind gemeinsam mit einem Bildschirm zur Steuerung an dem Medienrechner angeschlossen, wober der Steuerungsbildschirm an die Grafikkarte angeschlossen werden muss.

(Abb. Video-Ins)

\subsubsection{Software}

Der Medienrechner steuert die Projektoren mit dem Programm Resolume an. Um die Wiedergabe zu starten muss in diesem ein Doppelklick auf dem Reiter über den Video-Vorschaubildern ausgeführt werden.

(Abb. Resolume)

\section{Raum 2 ('Platine')}

\section{Raum 3 ('Server')}

\section{Raum 4 ('Cloud')}

\end{document}